\documentclass[notitlepage]{article}
\begin{document}
Muller diagrams were automatically generated by a custom program in python that is available at: http://github.com/uriba/muller-chart

The algorithm receives two inputs, the hierarchy of strains, and, for every time point at which measurements were taken, the fraction of every mutation out of the population at that time point.
A strain is therefore characterized by the set of mutations that it posseses, but that its ancestor strain does not.

Given these two inputs the algorithm first, for every mutation at every time point, rounds it down to $0$ if its fraction is smaller than a predefined threshold of $0.005$, and given that at neither of the adjecant measurement points the fraction of that mutation is not higher.
Next, the algorithm checks, for every mutation, that at every measurement point its fraction out of the population is at least as high as the sum of the fractions of the mutations of its direct decendent strains, where the test is performed recursively.
In case this condition fails, the algorithm rounds up the fraction of the mutation to satisfy the constraint.

Finally, the algorithm sets, and possibly adds time points to indicate the times at which every mutation arises.
If mutation $A$ first appeared in a measurement taken at time $t_i$, then $A$'s initiation time is set to $t_{i-1}$.
For every decendent strain of $A$, if it also first appeared at time $t_i$, then its initiation time is set to $\frac{t_i+t_{i-1}}{2}$.
This procedure then continues recursively for the decendents of each strain whose initiation time was set.

At this point the input is being used to generate the muller diagram in the following manner:
At every time point, the $[0,1]$ interval is divided to intervals representing the fractions the different mutations occupy out of the population, such that if mutation $M$ has a frequency $f_M$ at the given time point, it will be assigned an interval $[a,b]$ such that $b-a=f_M$.
The division is done recursively, starting with the wild type (the ancestor of all the strains) being assigned the entire $[0,1]$ interval.
Then, given that mutation $M$ occupies the interval $[a,b]$, and that there are $n$ direct decendent strains to the strain containing $M$, the unique mutations of these decendent strains are given evenly spaced intervals in $[a,b]$, representing their frequencies, with the space between the intervals of two adjecant decendents being the frequency of $M$, $f_M$, minus the sum of the frequencies of all of its direct decendents, divided by $n+1$ (to account of the spaces needed between the first and last decendents and the boundaries of the interval assigned to $M$, namely, $a$ and $b$.
For example, if $N$ and $L$ are the only decendents of $M$ and at time point $t$ their fractions are $0.2$, $0.4$ and $0.75$ respectively, and $M$ is assigned the interval $[0.2,0.95]$ then $N$ and $L$ will be assigned the intervals $[0.25,0.45]$ and $[0.5,0.9]$ respectively, so that their intervals are spaced by $0.05$ from the boundaries of $M$'s interval and between each other.
The recursion then proceeds to the decendents of $M$ until the entire strain hierarchy is covered.

Once the intervals are determined for all time points, the algorithm generates the diagram by traversing the strains hierarchy tree.
For every strain, identified by its unique mutation, the algorithm draws two lines, one connecting the lower bounds of the intervals of its mutation in time, and the other connecting the upper bounds.
The area between these two lines is filled with the color assigned to that mutation.
The algorithm then recursively continues to the direct decendents to overlay their area etc.

Figure $XXX$ has been manually manipulated, rounding straight lines and aligning strains, to make it more visually appealing.
\end{document}
